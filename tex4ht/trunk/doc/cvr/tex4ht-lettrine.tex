%$Id: template.tex,v 1.1 2011/09/13 04:19:25 cvr Exp cvr $

\documentclass[a4paper]{article}
\usepackage{xspace,graphicx,blog}

\let\hlink\href
\begin{document}

\section{\TeX4ht Supports Lettrine}

\hlink{http://www.raphink.info}{Rapha\"el Pinson} has kindly
contributed the code that enables \texht to support \Verb=lettrine=
package. (\Verb=lettrine.sty= helps to typeset dropped capitals in a
\latex document.) This probably will solve the long standing problem
of drop capital support in \texht, even if all the options of
\Verb=lettrine= package are not supported at this time. Hope,
Rapha\"el will continue support to enhance \texht by adding all the
missing options of \Verb=lettrine=.

Only \Verb=ante= option is supported now apart from basic command
\Verb=\lettrine= with two arguments, the first being the character for
dropped capital and second, rest of characters in that string or
chosen to form what is usually called \Verb=lettrine line=.

Seven \texht hooks have been provided to configure \Verb=lettrine= as
given hereunder:
\begin{enumerate}
\item Before \Verb=\lettrine=
\item After  \Verb=\lettrine=
\item Before lettrine initial character
\item Between lettrine initial character and lettrine line
\item After lettrine line
\item Before lettrine ante
\item After lettrine ante
\end{enumerate}
Further hooks, namely, \Verb=\HlettrineChar= and
\Verb=\HlettrineString= have been defined to access initial character
and lettrine line respectively, in case, we want to further
re-configure \Verb=\lettrine=.  The default configuration is provided
below:
\begin{verbatim}
\Configure{lettrine}
   {\HCode{<span class="lettrine lettrine-\HlettrineChar">}}
   {\HCode{</span>}}
   {\HCode{<span class="lettrine-letter">}}
   {\HCode{</span><span class="lettrine-line">}}
   {\HCode{</span>}}
   {\HCode{<span class="lettrine-ante">}}
   {\HCode{</span>}}
\end{verbatim}
Users shall write their own custom \Verb=css= declarations to format the
dropcapital and lettrine line to suit their taste. The default
\Verb=css= declarations are:
\begin{verbatim}
\Css{.lettrine{float: left;
    line-height: 0.7; margin-left: -0.1em;
    margin-bottom: -.5em; margin-right: 0.2em;}}
\Css{.lettrine-ante{vertical-align: top;}}
\Css{.lettrine-letter{font-family: InitialsFont; font-style: normal;
    font-size: 4em; color: gray;}}
\Css{.lettrine-A{margin-right: 0.3em;}}
\Css{.lettrine-A + .lettrine-line{margin-left: -0.4em;}}
\Css{.lettrine-À{margin-right: 0.3em;}}
\Css{.lettrine-À + .lettrine-line{margin-left: -0.4em;}}
\Css{.lettrine-J{line-height: 1; margin-right: 0;}}
\Css{.lettrine-H, .lettrine-I, .lettrine-N, .lettrine-U{margin-right: 0;}}
\Css{.lettrine-V{margin-right: -0.3em;}}
\Css{.lettrine-V + .lettrine-line{margin-left: 0.3em;}}
\Css{.lettrine-Q{padding-bottom: 1em;margin-top: -0.6em;}}
\Css{.lettrine-line{font-variant: small-caps;}}
\end{verbatim}

A typical \Verb=html= page generated with the help of this newly
created
\hlink{http://download.river-valley.com/tex4ht/lettrine/lettrine.4ht}
{\Verb=lettrine.4ht=} can be seen
\hlink{http://download.river-valley.com/tex4ht/lettrine/ltrn-test.html}{here}.

\end{document}


